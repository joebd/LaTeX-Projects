\documentclass{beamer}
\usetheme{cambridgeus}
\usepackage{graphicx}
\usepackage{amsmath}
\usepackage{amsthm}
\usepackage{relsize}
\newtheorem*{h2}{(H_2)}
\newtheorem*{h3}{(H_3)}

\begin{document}

%% TITLE PAGE

\title{Political Party Competition and Campaign Finance Laws}
\author{\author{Joe Blaszcynski}}
\begin{frame}
\titlepage
\date{??}
\end{frame}

%% OVERVIEW

\begin{frame}
\frametitle{Overview}
\begin{itemize} 
\item{Background} \\~\
\item{Synopsis} \\~\
\item{Replication Methodology} \\~\
\item{Research Proposal} \\~\  
\item{Methodology} \\~\ 
\end{itemize}
\end{frame}

%% INTRODUCTION

\begin{frame}
\frametitle{Introduction} 
\begin{itemize}
\item{Does campaign finance laws influence political party competition?} 
\end{itemize}
	\begin{enumerate}[i]
	\item{Potter, Joshua D. and Margit Tavits. 2015. "The Impact of Campaign Finance Laws on Party Competition." \emph{British Journal of Political Science} 45(1): 73-95.} \\~\
\end{enumerate}
\begin{itemize}
	\item{Comparative Perspective} 
	\item{Democratic elections}
	\item{Campaign Finance}
\end{itemize}
\end{frame}

%% SYNOPSIS 

\begin{frame} 
\frametitle{Synopsis}
\begin{itemize}
	\item{%WHAT THE STUDY IS ABOUT}
\end{itemize}
\end{frame}


%% HYPOTHESIS

\begin{frame}
  \frametitle{Hypotheses}
  \begin{block}{}
   		 \textbf{(H1)}: \emph{Where fund parity is high, more fractionalized party systems with a higher effective number of parties}\\~\
\end{block}
\end{frame}

%% REPLICATION METHODOLOGY

\begin{frame}
\frametitle{Replication Methodology}
\textbf{Regression Analysis (OLS)} 
\begin{itemize}
	\item{%% something}\\~\
	\end{itemize}
\textbf{Methods} 
\begin{itemize}
	\item{Quasi-Experimental} 
	\end{itemize}
		\begin{enumerate}[i]
		\item{{Treatment and Control effects}}
\end{enumerate}
\end{frame}

%% Research Proposal 

\begin{frame}
\frametitle{Research Proposal} 
\section{Methodology}
\begin{itemize}
	\item {Treatment and Control Effects (DiD)}
\end{itemize}
\begin{itemize} 
	\item{Comparative analysis on democratic countries}\\~\
	\item{Another point}\\~\
\end{itemize}
\end{frame}


\setlength{\tabcolsep}{20 pt}
\renewcommand{\arraystretch}{1.5}

\begin{frame} {Test title for the table}
	\begin{table}
	\centering
	\begin{tabular}{lcc}
		& All democracies & 1974 and later \\
		\hline\hline
			Fund Parity & 0.44$^{***}$ & 0.45$^{**}$ \\
			& (0.15) & (0.21) \\ 
			Democratic Years & 0.01 & -0.01 \\
			& (0.01) & (0.03) \\
			Federal &  -0.21 & -0.23 \\
			& (0.48) & (0.75) \\ 
			Presidential & -0.17 & -0.03 \\
			& (0.21) & (0.28) \\
			District Magnitude (Logged) & 0.60$^{**}$ & 0.34 \\
			& (0.30) & (0.45) \\
			Ethnolinguistic Fractionalization & 0.96 & -0.43 \\
			& (1.29) & (1.91) \\
			Magnitude x Fractionalization & 0.75 & -0.58 \\
			& (0.64) & (0.89) \\
			Intercept & 3.07$^{***}$ & 4.38$^{***}$ \\ 
			& (0.76) & (1.23) \\~\
			N & 90 & 54 \\ 
			R$^{2}$ & 0.20 & 0.17 \\ 
				\bottomrule
					\textit{Note:} ${*}$\emph{p}<0.1, ${**}$\emph{p}<0.05, ${***}$\emph{p}<0.01. 
					
		\end{tabular}
	\end{table}
\end{frame} 

\end{document}



